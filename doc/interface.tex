\pagebreak
	\section{Interface}
	\label{chap:interfaceLang}

		\subsection{GNU and Intel Fortran}

		An interface example fortran program is included in \texttt{interface/fortGrid2Grid}. To communicate with \texttt{Grid2Grid}, the fortran script \texttt{modCommG2G.f90} in \texttt{interface/}\texttt{fortGrid2Grid} is needed. It is a communication module with \texttt{libGrid2Grid.so}. The fortran interface example program is the following :

		\begin{lstlisting}[language={[95]Fortran}]

	!! Program Start ----------------------------------------------------
	Program Main
	!! ------------------------------------------------------------------
	use modCommG2G			!! Use Communication Module
	!! ------------------------------------------------------------------
	Implicit None
	!! Variables --------------------------------------------------------
	Integer,Parameter      :: nChar = 300			!! Default Character Length
	Character(len = nChar) :: grid2gridPath		!! libGrid2Grid.so Path

	integer                :: hosIndex				!! HOS Index
	Character(len = nChar) :: hosSolver				!! HOS Solver (Ocean or NWT)
	Character(len = nChar) :: hosFileName			!! HOS Result File Path

	Double precision       :: zMin, zMax			!! Surf2Vol Domain
	integer                :: nZmin, nZmax		!! Number of vertical grid
	Double precision       :: zMinRatio, zMaxRatio	!! Grading ratio (=3)

	Double precision       :: t, dt			!! Simulation Time, dt
	Double precision       :: x, y, z		!! Computation Point
	Double precision       :: eta, u, v, w, pd	!! HOS Wave Information
	!! Dummy variables --------------------------------------------------
	integer                :: it				!! Dummy time loop integer

	!! Program Body -----------------------------------------------------

	!!!... Write Program Start
	write(*,*) "Test program (Connect to Fortran) to use Grid2Grid shared library"

	!!!... Set libGrid2Grid.so path.
	!!!    It is recommended to use absolute path
	! grid2gridPath = "/usr/lib/libGrid2Grid.so"	(if soft link is made)
	grid2gridPath = "../../obj/libGrid2Grid.so"

	!!!... Load libGrid2Grid.so and connect subroutines
	Call callGrid2Grid(grid2gridPath)

	!!!... Declare HOS Index
	hosIndex = -1

	!!!... Set HOS Type (Ocean or NWT)
	hosSolver = "NWT"

	!!!... Set HOS Result file Path
	hosFileName = "modes_HOS_SWENSE.dat"

	!!!... Set HOS Surf2Vol Domain and Vertical Grid
	zMin = -0.6d0; 				zMax =  0.6d0
	nZmin = 50; 					nZmax = 50
	zMinRatio = 3.d0; 		zMaxRatio = 3.d0

	!!... Initialize Grid2Grid and Get HOS Index
	Call initializeGrid2Grid(hosSolver, hosFileName, zMin, zMax, nZmin, nZmax, zMinRatio, zMaxRatio, hosIndex)

	!! Time Information
	t  = 0.0d0; 		dt = 0.1d0

	!! Given Point
	x = 0.5d0; 			y = 0.5d0; 			z = -0.5d0

	!! Time Loop
	do it = 1,10

		!! Correct HOS Vol2VOl for given time
		Call correctGrid2Grid(hosIndex, t)

		!! Get Wave Elevation
		Call getHOSeta(hosIndex, x, y , t, eta)

		!! Get Flow Velocity
		Call getHOSU(hosIndex, x, y, z, t, u, v ,w)

		!! Get Dynamic Pressure
		Call getHOSPd(hosIndex, x, y, z, t, pd)

		!! Write Flow Information
		write(*,*) t, eta, u, v, w, pd

		!! Time Update
		t = t + dt
	enddo

	!! Write End of Program
	write(*,*) "Test program (Connect to Fortran) is done ..."
	!! ------------------------------------------------------------------
	End Program
	!! ------------------------------------------------------------------
		\end{lstlisting}

		\pagebreak
		\subsection{OpenFOAM}

		An interface example OpenFOAM program is included in \texttt{interface/ofGrid2Grid}. The shared library \texttt{libGrid2Grid.so} should be compiled at \texttt{\$FOAM\_USER\_LIBBIN}. To check \texttt{libGrid2Grid.so} exists at \texttt{\$FOAM\_USER\_LIBBIN}, use following shell command :

		\begin{lstlisting}[language=bash]

		$	ls $FOAM_USER_LIBBIN/libGrid2Grid.so
		\end{lstlisting}

		If \texttt{libGrid2Grid.so} not exists, refer to Chapter \ref{chap:Grid2GridInstall}.

		To call shared library \texttt{libGrid2Grid.so} in \texttt{\$FOAM\_USER\_LIBBIN}, OpenFOAM compiling option is added at \texttt{Make/option}. Open \texttt{Make/option} and add following compiling option.

		\begin{lstlisting}[language=bash]

		EXE_LIBS = \
							...
							-lgfortran \
							-L$(FOAM_USER_LIBBIN) \
							-lGrid2Grid
		\end{lstlisting}

		OpenFOAM interface example program is given next page.

		\pagebreak

		\begin{lstlisting}[language=c++]

#include "fvCFD.H"

namespace Foam
{
	//- Grid2Grid Initial Character Length
	const int nCharGridGrid(300);

	//- Initialize Grid2Grid Class in Fortran
	//
	//  __modgrid2grid_MOD_initializegrid2grid
	//  (
	//      hosSolver,
	//      hosFileName,
	//      zMin,
	//      zMax,
	//      nZmin,
	//      nZmax,
	//      zMinRatio,
	//      zMaxRatio,
	//      hosIndex
	//  )
	//
	//    Input
	//      hosSolver            : "NWT" or "Ocean"
	//      hosFileName          : filePath of HOS mode result file
	//      zMin, zMax           : HOS grid zMin and zMax
	//      nZmin, nZmax         : HOS number of z grid
	//      zMinRatio, zMaxRatio : HOS z grid max/min ratio
	//
	//    Output
	//      hosIndex             : HOS Vol2Vol Index
	//
	extern "C" void __modgrid2grid_MOD_initializegrid2grid
	(
	const char[nCharGridGrid],
	const char[nCharGridGrid],
	const double*,
	const double*,
	const int*,
	const int*,
	const double*,
	const double*,
	int*
	);

	//- Correct Grid2Grid for given simulation Time
	//
	//  __modgrid2grid_MOD_correctgrid2grid(hosIndex, simulTime)
	//
	//    Input
	//      hosIndex   : HOS Vol2Vol Index
	//      simulTime  : Simulation Time
	//
	extern "C" void __modgrid2grid_MOD_correctgrid2grid
	(
	const int *,
	const double *
	);

	//- Get HOS Wave Elevation
	//
	//  __modgrid2grid_MOD_gethoseta(hosIndex, x, y, t, eta)
	//
	//    Input
	//      hosIndex : HOS Vol2Vol Index
	//      x, y, t  : (x and y) position and simulation Time (t)
	//
	//    Output
	//      eta      : wave elevation
	//
	extern "C" void __modgrid2grid_MOD_gethoseta
	(
	const int *,
	const double *,
	const double *,
	const double *,
	double *
	);

	//- Get HOS Flow Velocity
	//
	//  __modgrid2grid_MOD_gethosu(hosIndex, x, y, z, t, u, v, w)
	//
	//    Input
	//      hosIndex    : HOS Vol2Vol Index
	//      x, y, z, t  : (x, y, z) position and simulation Time (t)
	//
	//    Output
	//      u, v, w     : (x, y, z) - directional flow velocity
	//
	extern "C" void __modgrid2grid_MOD_gethosu
	(
	const int *,
	const double *,
	const double *,
	const double *,
	const double *,
	double *,
	double *,
	double *
	);
	//- Get HOS Dynamic Pressure
	//
	//  __modgrid2grid_MOD_gethospd(hosIndex, x, y, z, t, pd)
	//
	//    Input
	//      hosIndex    : HOS Vol2Vol Index
	//      x, y, z, t  : (x, y, z) position and simulation Time (t)
	//
	//    Output
	//      pd      : Dynamic Pressure p = -rho*d(phi)/dt-0.5*rho*|U*U|
	//
	extern "C" void __modgrid2grid_MOD_gethospd
	(
	const int *,
	const double *,
	const double *,
	const double * ,
	const double *,
	double *
	);

	//- Get HOS Wave Elevation, Flow Velocity and Dynamic Pressure
	//
	//  __modgrid2grid_MOD_gethosflow(hosIndex, x, y, z, t, eta, u, v, w, pd)
	//
	//    Input
	//      hosIndex    : HOS Vol2Vol Index
	//      x, y, z, t  : (x, y, z) position and simulation Time (t)
	//
	//    Output
	//      eta         : wave elevation
	//      u, v, w     : (x, y, z) - directional flow velocity
	//      pd          : Dynamic Pressure p = -rho * d(phi)/dt - 0.5 * rho * |U * U|
	//
	extern "C" void __modgrid2grid_MOD_gethosflow
	(
	const int *,
	const double *,
	const double *,
	const double * ,
	const double *,
	double *,
	double *,
	double * ,
	double *,
	double *
	);

}

// Main OpenFOAM Program Start
int main(int argc, char *argv[])
{

	// Write Program Start
	Info << "OpenFOAM Program Example to Call Grid2Grid (HOS Wrapper) in OpenFOAM" << endl;

	// Set HOS Solver Type
	const word HOSsolver_("NWT");
	const word HOSFileName_("./modes_HOS_SWENSE.dat");

	// Set File Name
	string strHOSSolver = string(HOSsolver_);
	string strHOSFileName = string(HOSFileName_);

	// Set HOS Solver Type
	const char *HOSsolver = strHOSSolver.c_str();

	// Set HOS Mode Result File Path
	const char *HOSfileName = strHOSFileName.c_str();

	// Set HOS Z Grid Information
	int indexHOS(-1);

	double zMin(-0.6), zMax(0.6);
	int nZmin(50), nZMax(50);

	double zMinRatio(3.0), zMaxRatio(3.0);

	// Initialize Grid2Grid
	__modgrid2grid_MOD_initializegrid2grid(HOSsolver, HOSfileName,
	&zMin, &zMax,
	&nZmin, &nZMax,
	&zMinRatio, &zMaxRatio, &indexHOS);

	Info << "HOS Label : " << indexHOS << endl;

	// Set Position
	double x(0.5), y(0.0), z(-0.5);

	// Define Flow Quantities
	double eta, u, v, w, pd;

	// Set Simulation Time and Time Difference
	double simulTime(0.0);
	double dt(0.1);



	// Time Loop
	for (int it = 0; it < 11; it++)
	{
		// Correct Grid2Grid
		__modgrid2grid_MOD_correctgrid2grid(&indexHOS, &simulTime);

		// Get Wave Eta
		__modgrid2grid_MOD_gethoseta(&indexHOS, &x, &y, &simulTime, &eta);

		// Get Flow Velocity
		__modgrid2grid_MOD_gethosu(&indexHOS, &x, &y, &z, &simulTime, &u, &v, &w);

		// Get Dynamic Pressure
		__modgrid2grid_MOD_gethospd(&indexHOS, &x, &y, &z, &simulTime, &pd);

		Info << " sumulTime : " << simulTime << endl;
		Info << "   eta     : " << eta << endl;
		Info << "   u, v, w : " << u << " " << v << " " << w << endl;
		Info << "   pd      : " << pd << nl << endl;

		// Get whole Information
		__modgrid2grid_MOD_gethosflow(&indexHOS, &x, &y, &z, &simulTime, &eta, &u, &v, &w, &pd);
		Info << "   eta     : " << eta << endl;
		Info << "   u, v, w : " << u << " " << v << " " << w << endl;
		Info << "   pd      : " << pd << nl << endl;

		// Time Update
		simulTime+=dt;
	}

	return 0;
}
		\end{lstlisting}
