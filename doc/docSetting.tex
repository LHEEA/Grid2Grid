%%
\documentclass[12pt]{article}
%%

%% Basic Setting -----------------------------------------------------------------------
\usepackage[english]{babel}
\usepackage{amsmath}		 		% math pack
\usepackage{mathtools}
\usepackage{url}					   % add urls
\usepackage[utf8x]{inputenc}	 % latex input utf8x (various characterers)
\usepackage{natbib}				     % reference package
\usepackage{parskip}				 % space between paragraph
\usepackage{fancyhdr}				% page headers and footers
\usepackage{vmargin}				% set various marfines for header/footer and page dimension
\usepackage{fixltx2e}				  % latex debugging package
%% ----------------------------------------------------------------------- Basic Setting

%% Table Setting -----------------------------------------------------------------------
\usepackage{array}						% array pack
\usepackage{caption}  				% caption Setup 
\usepackage{chngcntr}				% caption set to follow chapter number 
\usepackage{hhline}					% Seperate line properties (tabular)
\usepackage{multirow}				% multi row package
\usepackage{booktabs,caption}	% caption
\usepackage[flushleft]{threeparttable}	 % package for tabular notes

%%... Table Numbering and Format
\captionsetup[table]{labelfont=bf,  labelsep=period}	% table caption format
\captionsetup[table]{skip=5pt}		% space between caption and table
\counterwithin{table}{section}			% table caption (Chapter number)

%%... Table Column Alignment Variables
\newcolumntype{L}[1]{>{\raggedright\let\newline\\\arraybackslash\hspace{0pt}}m{#1}}
\newcolumntype{C}[1]{>{\centering\let\newline\\\arraybackslash\hspace{0pt}}m{#1}}
\newcolumntype{R}[1]{>{\raggedleft\let\newline\\\arraybackslash\hspace{0pt}}m{#1}}
%% ----------------------------------------------------------------------- Table Setting

%% Figure Setting ----------------------------------------------------------------------

\usepackage{graphicx}				% graph pack
\usepackage{subfig}					% sub figure
\usepackage{float}						% sub figure name
\graphicspath{{images/}}				% image path

%%... Figure format
\captionsetup[figure]{labelfont=bf,  labelsep=period}	% table caption format
\counterwithin{figure}{section}	% figure caption (Chapter number)
%% ---------------------------------------------------------------------- Figure Setting

%% Equation Setting ------------------------------------------------------------------
\counterwithin{equation}{section}		% equation caption (Chapter number)

%... frac font 
\newcommand\ddfrac[2]{\frac{\displaystyle #1}{\displaystyle #2}}
%% ------------------------------------------------------------------ Equation Setting

%% Auxillary -----------------------------------------------------------------------------
%\usepackage{subcaption}			% sub-caption 
%% ----------------------------------------------------------------------------- Auxillary


%% Page Setting ------------------------------------------------------------------------
\pagestyle{fancy}
\setmarginsrb{2.5 cm}{2 cm}{2.5 cm}{2 cm}{1 cm}{1.5 cm}{1 cm}{1.5 cm}
%% ------------------------------------------------------------------------ Page Setting

\usepackage{xcolor}
\usepackage{listings}             % Include the listings-package

\definecolor{codegreen}{rgb}{0,0.6,0}
\definecolor{codegray}{rgb}{0.5,0.5,0.5}
\definecolor{codepurple}{rgb}{0.58,0,0.82}
\definecolor{backcolour}{rgb}{0.95,0.95,0.92}

%\lstdefinestyle{customf}
%{
%	belowcaptionskip=1\baselineskip,
%	breaklines=true,
%	xleftmargin=0pt,
%	language={[95]Fortran},
%	showstringspaces=false,
%	basicstyle=\footnotesize\ttfamily,
%	keywordstyle=\color{blue},
%	commentstyle=\color{codegreen},
%	%  commentstyle=\itshape\color{red},
%	identifierstyle=\color{black},
%	stringstyle=\color{orange},
%	columns=fullflexible
%}

\lstdefinestyle{customf}
{
	backgroundcolor=\color{backcolour},
	commentstyle=\color{codegreen},
	keywordstyle=\color{blue},
	identifierstyle=\color{black},
	breakatwhitespace=false,
	basicstyle=\footnotesize\ttfamily,
	breaklines=true,
	captionpos=b,
	keepspaces=true,
	showspaces=false,
	showstringspaces=false,
	showtabs=false,
	tabsize=2
}

% Set your language (you can change the language for each code-block optionally)
\lstset{language={[95]Fortran},style=customf}
\newcommand{\includecode}{\lstinputlisting}

\usepackage[skip=0pt]{caption}

\usepackage{tcolorbox}

\tcbuselibrary{listings}
\newtcblisting{ccode}
{
	colframe = black,
	boxrule=0.5pt,
	listing only,
	listing options={language=C++}
}

\newtcblisting{shcode}
{
	colframe = black,
	boxrule=0.5pt,
	listing only,
	listing options={language=bash}
}

\usepackage{forest}

\definecolor{folderbg}{RGB}{124,166,198}
\definecolor{folderborder}{RGB}{110,144,169}

\def\Size{4pt}
\tikzset{
	folder/.pic={
		\filldraw[draw=folderborder,top color=folderbg!50,bottom color=folderbg]
		(-1.05*\Size,0.2\Size+5pt) rectangle ++(.75*\Size,-0.2\Size-5pt);  
		\filldraw[draw=folderborder,top color=folderbg!50,bottom color=folderbg]
		(-1.15*\Size,-\Size) rectangle (1.15*\Size,\Size);
	}
}


\usepackage{xcolor}