	\pagebreak
	\subsection{typVol2Vol}

	\subsubsection{Description}

	\texttt{Vol2Vol} interpolates the wave fields and return wave properties at given time and position. It has the multiple \texttt{Surf2Vol} classes and interpolation class. The \texttt{Vol2Vol} class structure is described in Fig \ref{fig:vol2volStructure}. The reconstructed HOS wave fields (snapshot of wave field) retrived from \texttt{Surf2Vol} are used to construct interpolation data. 
	
	The HOS result file contains the whole simulation time information. If the HOS wave fields and interpolation is applied for the whole HOS simulatio time, the pre-computation time becomes too long and it needs hugh computation memory. \texttt{Grid2Grid} is designed for the efficient wave-field reconstruction demanded by CFD solvers for a relatively short period and small domain. Therefore it is not necessary to reconstruct whole HOS wave fields. For the efficient memory control, the revolving algorithm is introduced to holds multiple HOS wave fields and to update HOS wave field if it is demanded.

	\vspace{1em}
	{
		\begin{figure} [H]
			\centering
			\includegraphics[scale=0.72]{images/c1.structure/"Vol2VolClassStructure".png}
			\vspace{0.2cm}
			\caption{\texttt{Vol2Vol} class structure}
			\label{fig:vol2volStructure}
		\end{figure}
	}

	\pagebreak

	When \texttt{Vol2Vol} initialised, the \texttt{Surf2Vol} class array is allocated and initialise multiple \texttt{Surf2Vol} and connect those \texttt{Surf2Vol} class array with the revolving algorithm. The revolving algorithm determine which \texttt{Surf2Vol} should be updated and the order of \texttt{Surf2Vol} array for the efficient generation of interpolation data. 
	
	The subroutine \texttt{correct} with an input $t$ first determine HOS \texttt{Surf2Vol} correction indexand \texttt{Surf2Vol} order by revolving algorithm. Only necessary HOS \texttt{Surf2Vol} is updated and re-ordered to constructed for interpolation data. The interpolation data is constructed from the multiple wave field snapshot. The interpolation class holds the intepolation classes specific to HOS and it communicates with the \texttt{bspline-Fortran} module.

	The subroutine \texttt{getEta}, \texttt{getU}, \texttt{getPd} and \texttt{getFlow} return the interpolated wave properties for the given time and space. 

	\subsubsection{Class (Type)}
	
	\textbf{Class} : \texttt{Vol2Vol}
	
	\hspace{0.5 cm} -- Data :
	
	\hspace{1.0 cm} $\circ$ \texttt{nInterp\_} : Interpolation order (2 : Linear, 3 : Quadratic, 4 : Cubic, ... )
	
	\hspace{1.0 cm} $\circ$ \texttt{nSaveT\_} : Number of Surf2Vol wave fields  (\texttt{nSaveT\_}  $>$ \texttt{nInterp\_})
	
	\hspace{1.0 cm} $\circ$ \texttt{HOSs2v\_(:)} : Array of \texttt{Surf2Vol} class
	
	\hspace{1.0 cm} $\circ$ \texttt{itp2D\_} : Interpolation class for 2D waves
	
	\hspace{1.0 cm} $\circ$ \texttt{itp3D\_} : Interpolation class for 3D waves
	
	\vspace{0.5cm}
	
	\hspace{0.5 cm} -- Functionality :
	
	\hspace{1.0 cm} $\circ$ \texttt{initialize} : Initialise HOS Vol2Vol class with HOS type, result file path, ...
	
	\hspace{1.0 cm} $\circ$ \texttt{correct} : Update HOS wave fields with real-time
	
	\hspace{1.0 cm} $\circ$ \texttt{getEta} : Get interpolated wave elevation 
	
	\hspace{1.0 cm} $\circ$ \texttt{getU} : Get interpolated wave velocity 
	
	\hspace{1.0 cm} $\circ$ \texttt{getPd} : Get interpolated dynamic pressure
	
	\hspace{1.0 cm} $\circ$ \texttt{getFlow} : Get flow information
	
	\hspace{1.0 cm} $\circ$ \texttt{destroy} : Class destroyer\\    
	
	\pagebreak
	
	\subsubsection{How to use}
	
	\hspace{0.5 cm} -- Initialise \texttt{Vol2Vol}
	
	\begin{lstlisting}[language={[95]Fortran}]
	
	Call hosV2V%initialize(hosType, filePath, zMin, zMax, nZmin, nZmax, zMinRatio, zMaxRatio, iflag)
	
	!	hosV2V				: Vol2vol Class (Type)
	!	hosType				: HOS Type (Ocean or NWT)
	!	filePath			: HOS result file path (modes_HOS_SWENS.dat)
	!	zMin, zMax 		: HOS grid z-minimum and z-maximul (vertical domain)
	!	nZmin, nZmax 	: Number of z-directional Grid	
	!
	!	zMinRatio, zMaxRatio  (Optional)
	!	: Ratio of maximum and minimum height of grid (default=3)
	!
	!	iflag : Wriging option (iflag = 1, write Grid2Grid information)
	
	! or 
	
	Call hosV2V%initialize(dict)
	
	!	hosV2V			:  Vol2vol Class (Type)
	!	dict				: HOS dictionary to initialize HOS vol2vol (Type)
	\end{lstlisting}
	
	\hspace{0.5 cm} -- Correct \texttt{Vol2Vol}
	
	\begin{lstlisting}[language={[95]Fortran}]
	
	Call hosV2V%correct(simulTime)
	
	!	hosV2V				: Vol2vol Class (Type)
	!	simulTime			: Simulation time (real time value)
	\end{lstlisting}	
	
	\hspace{0.5 cm} -- Get wave elevation from \texttt{Vol2Vol}
	
	\begin{lstlisting}[language={[95]Fortran}]
	
	eta = hosV2V%getEta(x, y, simulTime, iflag)
	
	!	hosV2V				: Vol2vol Class (Type)
	!	x, y					: x and y position 
	!	simulTime 		: Simulation time (real time value)
	!
	!	eta						: Wave elevation
	!
	!	iflag (Optional)	
	!	: if iflag = 1, nondimensional x and y can be given (default = 0). 	
	\end{lstlisting}		
	
	\pagebreak	
	
	\hspace{0.5 cm} -- Get wave velocity from \texttt{Vol2Vol}
	
	\begin{lstlisting}[language={[95]Fortran}]
	
	Call hosV2V%getU(x, y, z, simulTime, u, v, w, iflag)
	
	!	hosV2V				: Vol2vol Class (Type)
	!	x, y, z				: x, y and z position 
	!	simulTime 		: Simulation time (real time value)
	!
	!	u, v, z				: Wave velocity ( x, y, z )
	!
	!	iflag (Optional)	
	!	: if iflag = 1, nondimensional x and y can be given (default = 0). 	
	\end{lstlisting}	
	
	\hspace{0.5 cm} -- Get dynamic pressure from \texttt{Vol2Vol}
	
	\begin{lstlisting}[language={[95]Fortran}]
	
	pd = hosV2V%getPd(x, y, z, simulTime, iflag)
	
	!	hosV2V				: Vol2vol Class (Type)
	!	x, y, z				: x, y and z position 
	!	simulTime 		: Simulation time (real time value)
	!
	!	pd						: Dynamic velocity (pd = p - rho * g * z)
	!
	!	iflag (Optional)	
	!	: if iflag = 1, nondimensional x and y can be given (default = 0). 	
	\end{lstlisting}	

	\hspace{0.5 cm} -- Get flow information from \texttt{Vol2Vol}
	
	\begin{lstlisting}[language={[95]Fortran}]
	
	Call hosV2V%getFlow(x, y, z, simulTime, eta, u, v, w, pd, iflag)
	
	!	hosV2V				: Vol2vol Class (Type)
	!	x, y, z				: x, y and z position 
	!	simulTime 		: Simulation time (real time value)
	!
	!	eta						: Wave elevation
	!	u, v, z				: Wave velocity ( x, y, z )
	!	pd						: Dynamic velocity (pd = p - rho * g * z)
	!
	!	iflag (Optional)	
	!	: if iflag = 1, nondimensional x and y can be given (default = 0). 	
	\end{lstlisting}	
	
	\hspace{0.5 cm} -- Destroy \texttt{Vol2Vol}
	
	\begin{lstlisting}[language={[95]Fortran}]

	Call hosV2V%destroy()	
	\end{lstlisting}	