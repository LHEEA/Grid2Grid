\pagebreak
	\subsection{PostGrid2Grid}
	
	\label{chap:postGrid2Grid}

	\subsubsection{Description}
	
\texttt{PostGrid2Grid} is a HOS post-processing class. It generates 3D VTK files of wave fields for visualization and wave elevation time series computed from \texttt{Vol2Vol} class. Wave fields at desired simulation time and spatial domain and wave elevation time series can be re-generated at some provided wave probes position. 

\texttt{PostGrid2Grid} algorithm is depicted in Fig. \ref{fig:postGrid2GridAlgorighm}. \texttt{PostGrid2Grid} is initialised with input file. The input file \texttt{postGrid2Grid.dict} contains HOS grid information and post processing information which have dictionary format. \texttt{PostGrid2Grid} first reads and checks the input file and then build 3D visualization grid and wave probes. \texttt{Vol2Vol} class is also initialised. The subroutine \texttt{doPostProcessing} do time loop of \texttt{correct}. Subroutine \texttt{correct} first corrects the \texttt{Vol2Vol} class and gets the wave fields. If the grid option is set to no air mesh, 3D grid is fitted to wave elevation. It writes the results on files (3D VTK file and wave elevation time series).


	\vspace{0.2cm}
	
	{
		\begin{figure} [H]
			\centering
			\includegraphics[scale=0.78]{images/c1.structure/"postGrid2Grid_Algorithm".png}
			\vspace{0.5cm}
			\caption{\texttt{PostGrid2Grid} Algorithm}
			\label{fig:postGrid2GridAlgorighm}
		\end{figure}
	}	
	
	\pagebreak
	\subsubsection{Class(Type)}	
	
	\textbf{Class} : \texttt{PostGrid2Grid}
	
	\hspace{0.5 cm} -- Data :
	
	\hspace{1.0 cm} $\circ$ \texttt{hosVol2Vol\_} : \texttt{Vol2Vol} class
	
	\hspace{1.0 cm} $\circ$ \texttt{rectLGrid\_} : Rectangular linear grid for 3D wave fields (VTK output)
	
	\hspace{1.0 cm} $\circ$ \texttt{waveProbe\_(:)} : Wave probe 
	
	\vspace{0.5cm}
	
	\hspace{0.5 cm} -- Functionality (Public) :
	
	\hspace{1.0 cm} $\circ$ \texttt{initialize} : Initialise PostGrid2Grid class
	
	\hspace{1.0 cm} $\circ$ \texttt{correct} : Correct \texttt{Vol2Vol}, \texttt{rectLGrid\_} and \texttt{waveProbe\_} and write output
	
	\hspace{1.0 cm} $\circ$ \texttt{writeVTK} : Write 3D wave fields in VTK format
	
	\hspace{1.0 cm} $\circ$ \texttt{doPostProcessing} : Do post processing 
	
	\hspace{1.0 cm} $\circ$ \texttt{destroy} : Destuctor of PostGrid2Grid class
	
	\vspace{0.5cm}
	
	\hspace{0.5 cm} -- Functionality (Private) :
	
	\hspace{1.0 cm} $\circ$ \texttt{readPostG2GInputFileDict} : Read \texttt{PostGrid2Grid} input file
	
	\hspace{1.0 cm} $\circ$ \texttt{checkPostG2GParameter} : Check \texttt{PostGrid2Grid} input file
	
	\hspace{1.0 cm} $\circ$ \texttt{writeVTKtotalASCII} : Write wave fields (Including air domain)
	
	\hspace{1.0 cm} $\circ$ \texttt{writeVTKnoAirASCII} : Write wave fields (Grid is fitted to wave elevation)
	
	\hspace{1.0 cm} $\circ$ \texttt{writeWaveProbe} : Write wave elevation time series
	
	\pagebreak
	\subsubsection{Input File of PostGrid2Grid}
	
	\texttt{PostGrid2Grid} needs the input file. The input file name is \texttt{postGrid2Grid.dict}. The input is recognized by \texttt{dictionary(keyword + value with sub-dictionary)}. The special characters are treated as a comment (\texttt{!, \# and \textbackslash\textbackslash}) and C++ style comment block is also allowed. 
	
	\vspace{0.5cm}
	
	-- PostGrid2Grid variables 
	
	\begin{lstlisting}[language=bash]
	
	### Select HOS dictionary and set post processing option ----------------------------------- #
	
	Grid2Grid					HOSDict;	# HOS dictionary
	
	writeVTK					true;	# 3D visualization option; true or false
	
	writeWaveProbe		true;	# Wave probe option; true or false

	\end{lstlisting}
	
	
	-- HOS dictionary
	
	\begin{lstlisting}[language=bash]
	
	### HOS dictionary ----------------------------------- #
	
	HOSDict
	{
		type	Ocean;		# HOS solver type; NWT or Ocean
		
		filePath	example/modes_HOS_SWENSE.dat;	# HOS file path

		fileType	ASCII;	# Option; HOS result file type, ASCII or HDF5
		
		interpolationOrder	3;	# Option; 3:Cubic spline [Default]
		
		zMin		-0.6;		# HOS domain zMin		
		zMax		0.6;		# HOS domain zMax
		
		nZMin		50;		# HOS domain nZMin
		nZMax		50;		# HOS domain nZMax
		
		zMeshType	meshRatio;		# Mesh type (uniform, sine, meshRatio)
		
		zMinRatio	3.0;	# Option, z<=0 meshing ratio (meshRatio)
		zMaxRatio	3.0;	# Option, z<=0 meshing ratio (meshRatio)
				
		writeLog        true;	# Option; write log option	
	}
	\end{lstlisting}
	
	-- Simulation Dictionary
		
	\begin{lstlisting}[language=bash]
	
	### Simulation dictionary ------------------------------------------- #
	
	simulation
	{
		startTime		2712.0;
	
		endTime			2812.0;
	
		dt	     		0.1;
	}	
	\end{lstlisting}
	
	-- 3D visualization dictionary
	
	\begin{lstlisting}[language=bash]
	
	### VTK mesh dictionary ------------------------------ #
	
	vtkMesh
	{
		airMesh		false;		# Air meshing option 
		
		xMin		 0.0;			# VTK mesh xMin
		xMax		100.0;		# VTK mesh xMax

		yMin		 0.0;			# VTK mesh yMin
		yMax		100.0;		# VTK mesh yMax		
		
		nX			50;				# VTK mesh nX
		nY			50;				# VTK mesh nX
		
		zMin		-0.6;			# VTK mesh zMin		
		zMax		0.6;			# VTK mesh zMax
		
		nZMin		50;				# VTK mesh nZMin
		nZMax		50;				# VTK mesh nZMax
		
		zMeshType	meshRatio;		# Mesh type (uniform, sine, meshRatio)
		
		zMinRatio	3.0;	# Option, z<=0 meshing ratio (meshRatio)
		zMaxRatio	3.0;	# Option, z<=0 meshing ratio (meshRatio)
		
	}
	\end{lstlisting}
	
	\pagebreak
	
	-- Wave probe dictionary
	
	\begin{lstlisting}[language=bash]
	
	### Wave probe dictionary ------------------------------------------ #
	
	waveProbe
	{
		waveProbeFile	waveElevation.dat;		# Output file name
		
		waveProbes
		{
			probe1
			{
				position  ( 0.0   10.0 );		# wave probe position
			}			
			probe2
			{
				position  ( 0.0   15.0 );
			}			
		}
	}
	\end{lstlisting}
	
	\subsubsection{How to use}
	
	\vspace{0.5cm}
	
	\begin{lstlisting}[language={[95]Fortran}]	

	!! Subroutine to run PostGrid2Grid -----------------------------------
	Subroutine runPostGrid2Grid(inputFileName)
	!! -------------------------------------------------------------------
	Use modPostGrid2Grid					!! Use PostGrid2Grid Module
	!! -------------------------------------------------------------------
	Implicit none	
	Character(Len = * ),intent(in) :: inputFileName		!! postGrid2Grid.inp	
	Type(typPostGrid2Grid)         :: postG2G				!! postGrid2Grid Class
	!! -------------------------------------------------------------------	
	
	!! Initialize PostGrid2Grid with Input File	
	Call postG2G%initialize(inputFileName)
	
	!! Do Post Processing
	Call postG2G%doPostProcessing()
	
	!! Destroy
	Call postG2G%destroy
	
	!! -------------------------------------------------------------------
	End Subroutine
	!! -------------------------------------------------------------------	
	\end{lstlisting}